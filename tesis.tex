\documentclass[a4paper,12pt,notitlepage,spanish,final]{report} % Imprime por un solo lado de la hoja
%\documentclass[letterpaper,12pt,notitlepage,spanish,final,twoside]{report} % Imprime por los dos lados de la hoja
\usepackage{algorithm}
\usepackage[noend]{algorithmic}
\usepackage{amsfonts}
\usepackage[centertags]{amsmath}
\usepackage{amssymb}
\usepackage{amsthm}
\usepackage[spanish]{babel}
\usepackage[margin=12pt,font=small,labelfont=bf,labelsep=endash,skip=10pt]{caption}
\usepackage{dsfont}
\usepackage[dvips]{epsfig}
\usepackage[mathcal]{euscript}
\usepackage{float}
\usepackage{slashbox}
%\usepackage{graphics}
%\usepackage{graphicx}
\usepackage{hyperref}
\usepackage[latin1]{inputenc}	% ISO-8859-1 codificación en español (incluye acentos)
%\usepackage[utf8]{inputenx} %Codificacion del texto (ISO Latin1 encoding)
\usepackage{newlfont}
% Para incluir el Glosario. Notar que se debe compilar usando: "makeindex tesis.nlo -s nomencl.ist -o tesis.nls" (orden: pdflatex, makeindex y pdflatex)
\usepackage[intoc, refpage]{nomencl}	
\usepackage{subfig}
\usepackage{sty/utfsm_tesis}
\usepackage{upgreek}
\usepackage{url}
\usepackage[table]{xcolor}
\usepackage{xtocinc}		% Include Table of Contents as the first entry in TOC
\usepackage{listings}
\lstset{ %
  language=Java
  }
% ---------------------------------------------------------------------------------------------------------------
% Fuzz
\hfuzz 2pt
\hoffset -1.0in		% Seteo a 0 el margen izquierdo
%\oddsidemargin 4cm		% Margen izquierdo (pag. impar)
\oddsidemargin 3cm	% Ancho Legal 21,59cm
\evensidemargin 0.5cm	% Alto Legal 35,56cm
\textwidth 15.5cm
\topmargin -1.5cm
%\voffset 2cm 		% Margen superior
\textheight 22cm
%\parindent 0em
%\parskip 2ex
\newlength{\defbaselineskip}
\setlength{\defbaselineskip}{\baselineskip}
\newcommand{\setlinespacing}[1]
           {\setlength{\baselineskip}{#1 \defbaselineskip}}
\newcommand{\doublespacing}{\setlength{\baselineskip}
                           {1.3 \defbaselineskip}}
\newcommand{\singlespacing}{\setlength{\baselineskip}{\defbaselineskip}}
% ---------------------------------------------------------------------------------------------------------------
% Fórmulas matemáticas utilizadas
\newcommand{\bbbr}{\mathbb R}
\hyphenation{de-no-mi-na-das} \hyphenation{co-rres-pon-den}
\hyphenation{ha-bi-tual-men-te} \hyphenation{ins-tan-cias}
\hyphenation{con-glo-me-ra-do} \hyphenation{an-te-rior-men-te}
\hyphenation{co-rres-pon-dien-te} \hyphenation{ins-truc-cio-nes}
\hyphenation{am-bien-tes}  \hyphenation{re-pre-sen-ta-cion}
\hyphenation{pre-sen-cia}
% Teoremas
\theoremstyle{plain}
\newtheorem{hipot}{Hipótesis}[chapter]
\newtheorem{thm}{Teorema}[section]
\newtheorem{cor}[thm]{Corolario}
\newtheorem{lem}[thm]{Lema}
\newtheorem{prop}[thm]{Proposición}
\newtheorem{defn}{Definición}[section]
\theoremstyle{remark}
\newtheorem{rem}{Observación}[section]
\numberwithin{equation}{section}
\renewcommand{\theequation}{\thesection.\arabic{equation}}
\floatname{algorithm}{Algoritmo}
% ---------------------------------------------------------------------------------------------------------------
\setlength{\tclineskip}{1.05\baselineskip}
% ---------------------------------------------------------------------------------------------------------------
\newcommand{\anc}{5.1cm}
\newcommand{\alt}{5.1cm}
\newcommand{\M}{{\mathcal M}}
\newcommand{\ra}{\rightarrow}
\newcommand{\converge}[2]{\underset{#1 \ra #2}{\ra}}
\newcommand{\limite}[2]{\underset{#1 \ra #2}{\lim}}
\newcommand{\deriv}[2]{\frac{\partial{#1}}{\partial{#2}} }
\newcommand{\ve}{\varepsilon}
\newcommand{\V}{{\mathcal V}}
\newcommand{\E}{{\cal E}}
\newcommand{\norm}[1]{\left\Vert#1\right\Vert}
\newcommand{\abs}[1]{\left\vert#1\right\vert}
\newcommand{\eps}{\varepsilon}
\newcommand{\s}[1]{{\mathbf #1}}
\newcommand{\ol}{\overline}
\newcommand{\Real}{{\mathbb R}}
\newcommand{\C}{{\mathcal C}}
\newcommand{\W}{{\mathcal W}}
\newcommand{\D}{{\mathcal D}}
\newcommand{\minimo}[1]{\underset{#1}{\min}}
\newcommand{\thetaM}{\hat{\s{\theta}}_n^M}
\newcommand{\I}{\mathcal{I}}
\newcommand{\N}{\mathcal{N}}
\newcommand{\Hip}{\mathcal{H}}
\newcommand{\ms}[1]{\mathbf{#1}}
% ---------------------------------------------------------------------------------------------------------------
%Redefinición de comandos del paquete "algorithmic"
\renewcommand{\algorithmicwhile}{\textbf{Mientras}}
\renewcommand{\algorithmicfor}{\textbf{Para}}
\renewcommand{\algorithmicdo}{\textbf{hacer}}
\renewcommand{\algorithmicend}{\textbf{fin}}
% ---------------------------------------------------------------------------------------------------------------
% Redefinición de comandos del paquete "nomencl".
\renewcommand{\nomname}{Glosario}
\renewcommand{\pagedeclaration}[1]{. Ver página~\hyperpage{#1}.} % Usar este comando en conjunto con el paquete "hyperref"
\makeatletter % necesario para que reconozca a '@' como carácter normal 
\renewcommand{\paragraph}{\@startsection{paragraph}{4}{\z@}{-3.25ex \@plus 
-1ex \@minus -.2ex}{1.5ex \@plus .2ex}{\normalfont\normalsize\bfseries}} 
\makeatother % necesario para que restablezca '@' como carácter especial
\makenomenclature %Genera el Glosario
% ---------------------------------------------------------------------------------------------------------------
\DeclareGraphicsExtensions{.pdf,.png,.jpg}
\graphicspath{{./img/}}

% ---------------------------------------------------------------------------------------------------------------
\begin{document}
\begin{center}
\ing
\copyrightyear{2015}
\submitdate{\today}
\convocation{mes}{2015}
% ---------------------------------------------------------------------------------------------------------------
\title{An�lisis y Aplicacion de Tecnicas de Analisis Multicriterio Jerarquico y en Red}
\author{Victor Andres Roberto Gonzalez Rodriguez}
\end{center}

%\twosupervisors
\profguia{Lautaro Guerra}
\profcorr{Cecilia Reyes}

\ack{tex/00.0-agradecimientos} % Incluir Agradecimientos
\dedicate{`` "}	
\resumenesp{tex/00.1-resumen}			% Incluir Resumen (Español)
\resumening{tex/00.2-abstract}			% Incluir Abstract (Inglés)
\abreviaciones{tex/00.3-glosario}		% Incluir Glosario

% ---------------------------------------------------------------------------------------------------------------
\setcounter{tocdepth}{3}
\beforepreface

\afterpreface
% ---------------------------------------------------------------------------------------------------------------
%!TEX root = ../tesis.tex
% Introducci? de la Tesis
% ---------------------------------------------------------------------------------------------------------------
\chapter{Introducci?}
\label{cha:intro}

\section{Definici? del problema}
\label{sec:problematica}

\section{Objetivos}
\label{sec:objetivos}
\subsection{Objetivo principal}

El presente trabajo busca estudiar, proponer y comparar  modelos de selecci? de los productos m? eficientes del mercado, desarrollado por Topten International Group (TIG), la Superintendencia de Energ? y Combustibles del Gobierno de Chile (SEC) y Fundaci? Chile (FCh), el cual utiliza distintos criterios para cada categor? de producto que describen la eficiencia energ?ica de ellos. Para esto se recopilar?y analizar?la informaci? relevante en el ?ea de eficiencia energ?ica de productos dom?ticos, lo cual ser?realizado mediante un trabajo colaborativo con expertos en el ?ea de la energ? y eficiencia.

Luego, se construir? modelos con el objetivo de encontrar los productos m? eficientes para cada categor?, para mejorar y promover el consumo sustentable en los hogares chilenos mediante el impulso de pol?icas sociales y medioambientales impulsadas por la SEC y el Ministerio de Energ?, utilizando metodolog?s multicriterio, para mejorar y apoyar la selecci? de los productos de manera estrat?ica, escogiendo el modelo que satisfaga de mejor manera el objetivo propuesto, para finalmente comparar y dar a conocer las importantes diferencias entre los modelos.

\subsection{Objetivos espec?icos}
\begin{itemize}
\item Familiarizarse con los conceptos y lineamientos de la eficiencia energ?ica impulsadas por el Ministerio de Energ? y de la Superintendencia de Energ? y Combustibles.
\item Comprender e identificar los criterios utilizados para categorizar los productos m? eficientes, utilizando las propuestas realizadas por los actores involucrados en el proyecto.
\item Identificar y analizar los principales m?odos de decisi? multicriterio discretos, y determinar el que permita resolver de mejor forma el objetivo propuesto.
\item Utilizar la metodolog? AHP, o Proceso Anal?ico Jer?quico, para dise?r un modelo jer?quico, cuyo objetivo sea identificar los productos m? eficientes del mercado.
\item Utilizar la metodolog? ANP, o Proceso Anal?ico en Red, para dise?r un modelo en red, cuyo objetivo sea identificar los productos m? eficientes del mercado.
\item Evaluar y comparar los modelos generados, escogiendo el que mejor resuelva la problem?ica, entregando una herramienta que apoye y establezca un modelo confiable para poder seleccionar los productos m? eficientes del mercado.
\item Dar a conocer las diferencias conceptuales entre los modelos, y esclarecer las ventajas y desventajas de cada uno para cada caso en especial.
\end{itemize}

\section{Alcances}
\label{sec:alcances}

Los modelos desarrollados se validar? mediante la retroalimentaci? de expertos en el ?ea de la eficiencia energ?ica y sustentabilidad, y se realizar?una comparaci? entre los resultados, con el fin de identificar los modelos que se ajusten m? a los lineamientos pol?icos y sociales que se busca establecer con la selecci? de los productos m? eficientes.

\section{Estructura del documento}
\label{sec:estructura}

%Introducci?			\ref{ch:intro} 
%Estado del Arte		\ref{ch:eda} 
%Herramientas Actuales	\ref{ch:herram}
%Analisis comparativo	\ref{ch:analis}
%implementaciones		\ref{ch:implem}
%Conclusiones			\ref{ch:conc}
%Ap?dice A				\ref{ch:apeA}
%Ap?dice B				\ref{ch:xxxx}
% ---------------------------------------------------------------------------------------------------------------
% Estado del Arte
% ---------------------------------------------------------------------------------------------------------------
\definecolor{dkgreen}{rgb}{0,0.6,0}
\definecolor{gray}{rgb}{0.5,0.5,0.5}
\definecolor{mauve}{rgb}{0.58,0,0.82}

\lstset{frame=tb,
  language=Java,
  aboveskip=3mm,
  belowskip=3mm,
  showstringspaces=false,
  columns=flexible,
  basicstyle={\small\ttfamily},
  numbers=none,
  numberstyle=\tiny\color{gray},
  keywordstyle=\color{blue},
  commentstyle=\color{dkgreen},
  stringstyle=\color{mauve},
  breaklines=true,
  breakatwhitespace=true
  tabsize=3
}

\chapter{Estado del Arte}
\label{ch:eda}

En este cap�tulo se dar� a conocer el estado actual de la cartera energ�tica de Chile, en la cual se encuentra trabajando el gobierno de Chile en conjunto con distintas agencias gubernamentales y no-gubernamentales, espec�ficamente en el �rea de la Eficiencia Energ�tica. En conjunto con esto, se dar� a conocer las distintas metodolog�as de los procesos de toma de decisiones multicriterio que se utilizan en la actualidad.

\section{Referencia del Sector Energ�a}
El sector de energ�a es estrat�gico y fundamental para el funcionamiento de nuestra sociedad y la vida de las personas. La energ�a es una fuente necesaria para el uso de artefactos el�ctricos, de calefacci�n y cocina, as� como tambi�n para el transporte y el funcionamiento del sector productivo.


El contexto mundial y nacional de las tres �ltimas d�cadas es radicalmente distinto del
escenario que se proyecta para los pr�ximos treinta a�os. Los hidrocarburos (carb�n, petr�leo y gas) se presentaban hasta hace unos a�os como una fuente de energ�a abundante, barata y respuesta preferente a los desaf�os que el desarrollo econ�mico mundial requer�a. Sin embargo, la creciente urbanizaci�n mundial y la irrupci�n de nuevos pa�ses como grandes consumidores de energ�a, probablemente implicar� un panorama m�s complejo de escasez y alta competencia por el uso de algunos combustibles, mayor volatilidad y altos precios de la energ�a. Las emisiones de contaminantes locales y globales de los hidrocarburos son una raz�n adicional para disminuir la dependencia de los combustibles f�siles y buscar nuevas fuentes energ�ticas propias, m�s limpias y a precios accesibles

\subsection{Energ�a en Chile}
Chile importa el 60\% de su energ�a primaria (Balance Nacional de Energ�a BNE 2012), por lo que somos un pa�s subordinado a la inestabilidad y volatilidad de los precios en los mercados internacionales y las restricciones de abastecimiento que se produzcan por fen�menos pol�ticos, clim�ticos o de mercado.\\

Los �ltimos diez a�os en Chile han estado marcados por el corte de gas natural desde Argentina, severos y largos per�odos de sequ�a, dificultades en el otorgamiento de permisos ambientales, insuficiente entrada de proyectos y de nuevas empresas en el �rea de generaci�n y escasa inversi�n en infraestructura en ese mismo segmento y tambi�n en transmisi�n el�ctrica. Todo ello ha contribuido a sostener a lo largo de la �ltima d�cada condiciones de estrechez de oferta de suministro el�ctrico, con altos costos marginales y precios a cliente final que reflejan un desarrollo ineficiente del sistema, lo que se ha agravado en los �ltimos a�os.\\

En efecto, los precios de la energ�a el�ctrica han aumentado considerablemente en la �ltima d�cada. En 2006, el suministro el�ctrico para el pueblo chileno, comercios y peque�as empresas (clientes regulados) fue adjudicado a valores promedio de US\$ 65 por MWh; en cambio, la �ltima licitaci�n, realizada en diciembre de 2013 para estos mismos clientes, fue adjudicada al doble del 2006 (valor promedio de US\$ 128 por MWh). Esto ha significado que la cuenta el�ctrica que pagan hoy las familias chilenas es un 20\% superior respecto al a�o 2010. De mantenerse el escenario de precios adjudicados en 2013, el costo de la electricidad podr�a subir otro 34\% durante la pr�xima d�cada.\\

Asimismo, en los �ltimos diez a�os, las industrias (clientes libres) han visto duplicados los precios por sus consumos el�ctricos, lo que resta competitividad a nuestra econom�a e impacta directamente en el crecimiento del PIB. En el a�o 2013, los precios medios de mercado rondaron en el Sistema Interconectado Central (SIC) los US\$ 112 por MWh y en el Sistema Interconectado del Norte Grande (SING) los US\$ 108 por MWh. La industria chilena est� enfrentando uno de los precios m�s altos de la energ�a el�ctrica en Am�rica Latina. En el caso de la miner�a, el sector enfrenta el segundo precio m�s alto con respecto a los pa�ses mineros a nivel mundial, y el doble con respecto a competidores directos, como Per�.

\subsection{La Eficiencia Energ�tica (EE) en Chile}

\subsection{Ministerio de Energ�a}

\subsection{Innovaci�n y Desarrollo Tecnol�gico en EE en Chile}

\subsubsection{Fundaci�n Chile}

\subsubsection{Superintentencia de Electricidad y Combustibles (SEC)}


\section{Toma de decisiones Multicriterio y Metodolog�a}

\subsection{Conceptos}

\subsection{Metodolog�a AHP}

\subsection{Metodolog�a ANP}
% Cap�tulo 3: Herramientas Actuales
% ---------------------------------------------------------------------------------------------------------------
\chapter{Marco Metodol�gico}
\label{ch:herram}

\section{Metodolog�a para el modelamiento e implementaci�n de las soluciones}

\section{Herramientas de Modelamiento}
\subsection{Herramientas de Proceso Anal�tico Jer�rquico (AHP)}
\subsection{Herramientas de Proceso Anal�tico en Red (ANP)}
% Cap�tulo 4: An�lisis Comparativo
% ---------------------------------------------------------------------------------------------------------------
\chapter{Modelamiento}
\label{ch:analis}

\section{Motivaci�n}

\section{Objetivo de los Modelos}
\subsection{Ampolletas}
\subsection{Refrigeradores}
\subsection{Aire Acondicionado}
\subsection{Electrodom�sticos}
\subsection{Autom�viles}
\subsection{Televisores}

\section{Modelos Generados}
\subsection{Ampolletas}
\subsection{Refrigeradores}
\subsection{Aire Acondicionado}
\subsection{Electrodom�sticos}
\subsection{Autom�viles}
\subsection{Televisores}
% Cap�tulo 5: Implementaci�n
% ---------------------------------------------------------------------------------------------------------------
\chapter{An�lisis Comparativo}
\label{ch:implem}

\section{Modelos Finales}
\subsection{Ampolletas}
\subsection{Refrigeradores}
\subsection{Aire Acondicionado}
\subsection{Electrodom�sticos}
\subsection{Autom�viles}
\subsection{Televisores}

\section{Evaluaci�n y An�lisis de los Criterios}

\section{An�lisis y Conclusiones}


%!TEX root = ../tesis.tex
% Cap?ulo 5: Conclusiones y Trabajo Futuro
% ---------------------------------------------------------------------------------------------------------------
\chapter{Conclusiones}
\label{ch:conc}
\section{Conclusiones Generales}

\section{Conclusiones Espec?icas}
\begin{itemize}
\item item
\end{itemize}


\section{Trabajo Futuro}

% ---------------------------------------------------------------------------------------------------------------

\appendix
%!TEX root = ../tesis.tex
% Ap?dice A: 
% ---------------------------------------------------------------------------------------------------------------
%\chapter{Ap?dice A}
%\label{ch:apeA}

%\section{Secci? Ap?dice}
%\label{sec:SecApe}

% ---------------------------------------------------------------------------------------------------------------
% ---------------------------------------------------------------------------------------------------------------
\singlespacing
\bibliographystyle{plain}
\bibliography{manual} 
% ---------------------------------------------------------------------------------------------------------------
\end{document} 