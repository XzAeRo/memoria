%!TEX root = ../tesis.tex
% Introducci? de la Tesis
% ---------------------------------------------------------------------------------------------------------------
\chapter{Introducci?}
\label{cha:intro}

\section{Definici? del problema}
\label{sec:problematica}

\section{Objetivos}
\label{sec:objetivos}
\subsection{Objetivo principal}

El presente trabajo busca estudiar, proponer y comparar  modelos de selecci? de los productos m? eficientes del mercado, desarrollado por Topten International Group (TIG), la Superintendencia de Energ? y Combustibles del Gobierno de Chile (SEC) y Fundaci? Chile (FCh), el cual utiliza distintos criterios para cada categor? de producto que describen la eficiencia energ?ica de ellos. Para esto se recopilar?y analizar?la informaci? relevante en el ?ea de eficiencia energ?ica de productos dom?ticos, lo cual ser?realizado mediante un trabajo colaborativo con expertos en el ?ea de la energ? y eficiencia.

Luego, se construir? modelos con el objetivo de encontrar los productos m? eficientes para cada categor?, para mejorar y promover el consumo sustentable en los hogares chilenos mediante el impulso de pol?icas sociales y medioambientales impulsadas por la SEC y el Ministerio de Energ?, utilizando metodolog?s multicriterio, para mejorar y apoyar la selecci? de los productos de manera estrat?ica, escogiendo el modelo que satisfaga de mejor manera el objetivo propuesto, para finalmente comparar y dar a conocer las importantes diferencias entre los modelos.

\subsection{Objetivos espec?icos}
\begin{itemize}
\item Familiarizarse con los conceptos y lineamientos de la eficiencia energ?ica impulsadas por el Ministerio de Energ? y de la Superintendencia de Energ? y Combustibles.
\item Comprender e identificar los criterios utilizados para categorizar los productos m? eficientes, utilizando las propuestas realizadas por los actores involucrados en el proyecto.
\item Identificar y analizar los principales m?odos de decisi? multicriterio discretos, y determinar el que permita resolver de mejor forma el objetivo propuesto.
\item Utilizar la metodolog? AHP, o Proceso Anal?ico Jer?quico, para dise?r un modelo jer?quico, cuyo objetivo sea identificar los productos m? eficientes del mercado.
\item Utilizar la metodolog? ANP, o Proceso Anal?ico en Red, para dise?r un modelo en red, cuyo objetivo sea identificar los productos m? eficientes del mercado.
\item Evaluar y comparar los modelos generados, escogiendo el que mejor resuelva la problem?ica, entregando una herramienta que apoye y establezca un modelo confiable para poder seleccionar los productos m? eficientes del mercado.
\item Dar a conocer las diferencias conceptuales entre los modelos, y esclarecer las ventajas y desventajas de cada uno para cada caso en especial.
\end{itemize}

\section{Alcances}
\label{sec:alcances}

Los modelos desarrollados se validar? mediante la retroalimentaci? de expertos en el ?ea de la eficiencia energ?ica y sustentabilidad, y se realizar?una comparaci? entre los resultados, con el fin de identificar los modelos que se ajusten m? a los lineamientos pol?icos y sociales que se busca establecer con la selecci? de los productos m? eficientes.

\section{Estructura del documento}
\label{sec:estructura}

%Introducci?			\ref{ch:intro} 
%Estado del Arte		\ref{ch:eda} 
%Herramientas Actuales	\ref{ch:herram}
%Analisis comparativo	\ref{ch:analis}
%implementaciones		\ref{ch:implem}
%Conclusiones			\ref{ch:conc}
%Ap?dice A				\ref{ch:apeA}
%Ap?dice B				\ref{ch:xxxx}
% ---------------------------------------------------------------------------------------------------------------