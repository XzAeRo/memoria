%!TEX root = ../tesis.tex
% Estado del Arte
% ---------------------------------------------------------------------------------------------------------------
\definecolor{dkgreen}{rgb}{0,0.6,0}
\definecolor{gray}{rgb}{0.5,0.5,0.5}
\definecolor{mauve}{rgb}{0.58,0,0.82}

\lstset{frame=tb,
  language=Java,
  aboveskip=3mm,
  belowskip=3mm,
  showstringspaces=false,
  columns=flexible,
  basicstyle={\small\ttfamily},
  numbers=none,
  numberstyle=\tiny\color{gray},
  keywordstyle=\color{blue},
  commentstyle=\color{dkgreen},
  stringstyle=\color{mauve},
  breaklines=true,
  breakatwhitespace=true
  tabsize=3
}

\chapter{Estado del Arte}
\label{ch:eda}

En este cap?ulo se dar?a conocer el estado actual de la cartera energ?ica de Chile, en la cual se encuentra trabajando el gobierno de Chile en conjunto con distintas agencias gubernamentales y no-gubernamentales, espec?icamente en el ?ea de la Eficiencia Energ?ica. En conjunto con esto, se dar?a conocer las distintas metodolog?s de los procesos de toma de decisiones multicriterio que se utilizan en la actualidad.

\section{Referencia del Sector Energ?}
El sector de energ? es estrat?ico y fundamental para el funcionamiento de nuestra sociedad y la vida de las personas. La energ? es una fuente necesaria para el uso de artefactos el?tricos, de calefacci? y cocina, as?como tambi? para el transporte y el funcionamiento del sector productivo.


El contexto mundial y nacional de las tres ?ltimas d?adas es radicalmente distinto del
escenario que se proyecta para los pr?imos treinta a?s. Los hidrocarburos (carb?, petr?eo y gas) se presentaban hasta hace unos a?s como una fuente de energ? abundante, barata y respuesta preferente a los desaf?s que el desarrollo econ?ico mundial requer?. Sin embargo, la creciente urbanizaci? mundial y la irrupci? de nuevos pa?es como grandes consumidores de energ?, probablemente implicar?un panorama m? complejo de escasez y alta competencia por el uso de algunos combustibles, mayor volatilidad y altos precios de la energ?. Las emisiones de contaminantes locales y globales de los hidrocarburos son una raz? adicional para disminuir la dependencia de los combustibles f?iles y buscar nuevas fuentes energ?icas propias, m? limpias y a precios accesibles

\subsection{Energ? en Chile}
Chile importa el 60\% de su energ? primaria (Balance Nacional de Energ? BNE 2012), por lo que somos un pa? subordinado a la inestabilidad y volatilidad de los precios en los mercados internacionales y las restricciones de abastecimiento que se produzcan por fen?enos pol?icos, clim?icos o de mercado.\\

Los ?ltimos diez a?s en Chile han estado marcados por el corte de gas natural desde Argentina, severos y largos per?dos de sequ?, dificultades en el otorgamiento de permisos ambientales, insuficiente entrada de proyectos y de nuevas empresas en el ?ea de generaci? y escasa inversi? en infraestructura en ese mismo segmento y tambi? en transmisi? el?trica. Todo ello ha contribuido a sostener a lo largo de la ?ltima d?ada condiciones de estrechez de oferta de suministro el?trico, con altos costos marginales y precios a cliente final que reflejan un desarrollo ineficiente del sistema, lo que se ha agravado en los ?ltimos a?s.\\

En efecto, los precios de la energ? el?trica han aumentado considerablemente en la ?ltima d?ada. En 2006, el suministro el?trico para el pueblo chileno, comercios y peque?s empresas (clientes regulados) fue adjudicado a valores promedio de US\$ 65 por MWh; en cambio, la ?ltima licitaci?, realizada en diciembre de 2013 para estos mismos clientes, fue adjudicada al doble del 2006 (valor promedio de US\$ 128 por MWh). Esto ha significado que la cuenta el?trica que pagan hoy las familias chilenas es un 20\% superior respecto al a? 2010. De mantenerse el escenario de precios adjudicados en 2013, el costo de la electricidad podr? subir otro 34\% durante la pr?ima d?ada.\\

Asimismo, en los ?ltimos diez a?s, las industrias (clientes libres) han visto duplicados los precios por sus consumos el?tricos, lo que resta competitividad a nuestra econom? e impacta directamente en el crecimiento del PIB. En el a? 2013, los precios medios de mercado rondaron en el Sistema Interconectado Central (SIC) los US\$ 112 por MWh y en el Sistema Interconectado del Norte Grande (SING) los US\$ 108 por MWh. La industria chilena est?enfrentando uno de los precios m? altos de la energ? el?trica en Am?ica Latina. En el caso de la miner?, el sector enfrenta el segundo precio m? alto con respecto a los pa?es mineros a nivel mundial, y el doble con respecto a competidores directos, como Per?.

\subsection{La Eficiencia Energ?ica (EE) en Chile}

\subsection{Ministerio de Energ?}

\subsection{Innovaci? y Desarrollo Tecnol?ico en EE en Chile}
En el sector de la Eficiencia Energ?ica en Chile, existes actores de car?ter p?blico, p?blico/privado y privado, que continuamente realizan proyectos relacionados con la Eficiencia Energ?ica y la Sustentabilidad, los cuales deben mantenerse alineados con los objetivos trazados por el Ministerio de Energ?, para as?aportar a la meta pa? y ademas mantenerse vigente en el mercado chileno.\\

Este sector se encuentra muy abandonado por el sector empresarial,   en donde existen en su mayor? PyMEs que se dedican al desarrollo de aplicaciones espec?icas para cada requerimiento de la industria. Es por eso que las empresas p?blico/privadas juegan un rol muy importante como gestor y vector de desarrollo de tecnolog?s, plataformas y m?ricas para la Eficiencia Energ?ica y Sustentabilidad, ya que estas empresas reunen de manera neutral los intereses de la industria junto a los del gobierno.\\

Por otro lado, los entes reguladores presentes en la realidad chilena, son

\subsubsection{Fundaci? Chile}

\subsubsection{Superintentencia de Electricidad y Combustibles (SEC)}


\section{Toma de decisiones Multicriterio y Metodolog?}

\subsection{Conceptos}

\subsection{Metodolog? AHP}

\subsection{Metodolog? ANP}